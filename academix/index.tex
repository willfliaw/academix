% ----------------------------------------------------------------------------
% index.tex
%
% This file defines the environment for the document's index.
% It supports automatic inclusion in the Table of Contents (TOC),
% handles balanced or unbalanced columns, and removes capital page marks.
% ----------------------------------------------------------------------------

% Boolean flag for determining if the index should be balanced across columns
\newboolean{balancedIndex}
\setboolean{balancedIndex}{true}

% Command to disable balanced index columns
\newcommand*\unbalancedIndex{\setboolean{balancedIndex}{false}}

% ----------------------------------------------------------------------------
% Custom index Environment
% ----------------------------------------------------------------------------
% A custom version of the `index` environment:
% - No capital page marks
% - Automatically included in the Table of Contents
% - Supports balanced and unbalanced columns for the index

\renewenvironment{index} {
    % Check if the document is in two-column mode, and temporarily switch to one-column
    \ifthenelse{\boolean{@twocolumn}}
    {\setboolean{restoreTwoColumnMode}{true}\onecolumn}
    {\setboolean{restoreTwoColumnMode}{false}}

    % Clear the page if needed based on the `openright` option
    \if@openright\cleardoublepage\else\clearpage\fi%

    % Set the page style for the index
    \thispagestyle{\chapterTitlePageStyle}
    \global\@topnum\z@ % Prevents figures from appearing at the top of the index page
    \@afterindentfalse % No indentation after headings
    \@mkboth{\indexname}{\indexname} % Set up running headers for the index

    % Add the index to the Table of Contents if `includeInTOC` is true
    \ifthenelse{\boolean{includeInTOC}}
    {\customAddcontentsLine{toc}{chapter}{\indexname}}
    {}

    \@makeSChapterHead{\indexname}
    \@afterheading

    \begingroup
    \setlength{\columnsep}{35pt}
    \setlength{\columnseprule}{0pt}
    \ifthenelse{\boolean{balancedIndex}}
    {\begin{multicols}{2}}
    {\begin{multicols*}{2}}
    \setSpacing{single}
    \setlength{\parindent}{0cm}
    \setlength{\parskip}{.3pt}
    \let\item\@idxitem % Define how index items are displayed
}{
    \ifthenelse{\boolean{balancedIndex}}
    {\end{multicols}}
    {\end{multicols*}}
    \endgroup

    % Restore two-column mode if it was originally enabled
    \ifthenelse{\boolean{restoreTwoColumnMode}}{\twocolumn}{}
}
