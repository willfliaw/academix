% ----------------------------------------------------------------------------
% page_style.tex
%
% This file defines the following:
%   - Font styles for header elements (rightmark, leftmark, page number)
%   - Custom page styles:
%       * "header" (with chapter and section marks)
%       * "plainheader" (plain header with page number)
%       * "ruledheader" (with underlined header)
%   - Indentation settings for the first paragraph of each section
%   - Sets the style for chapter titles
% ----------------------------------------------------------------------------

% ---------------------------------------------------------------------
% Font Styles for Header Elements
% ---------------------------------------------------------------------

\newcommand{\rightMarkFormat}{\small\itshape}   % Style for right header mark
\newcommand{\leftMarkFormat}{\itshape}          % Style for left header mark
\newcommand{\thePageFormat}{\small}             % Style for page number

% Custom commands for setting chapter and section marks in headers
\newcommand{\customMarkBoth}[2]{
    \ifthenelse{\boolean{nextOutOfTOC}}
    {\markboth{\nextMark}{\nextMark}}
    {\markboth{#1}{#2}}
}

% ---------------------------------------------------------------------
% Page Style: "header"
% ---------------------------------------------------------------------
% Defines a custom page style with headers for both odd and even pages

\newcommand{\ps@header}{
    \renewcommand{\@oddfoot}{}
    \renewcommand{\@evenfoot}{}
    \renewcommand{\@oddhead}{{\rightMarkFormat\rightmark}\hfill{\thePageFormat\thepage}}
    \renewcommand{\@evenhead}{{\thePageFormat\thepage}\hfill{\leftMarkFormat\leftmark}}

    % Ensure \chapter* shows the header
    \let\@mkboth\customMarkBoth

    % Define how \chapter marks the header
    \renewcommand{\chaptermark}[1]{
        \markboth{
            \ifnum \c@secnumdepth > \m@ne
                \thechapter\ \ %
            \fi
            ##1
        }{
            \ifnum \c@secnumdepth > \m@ne
                \thechapter\ \ %
            \fi
            ##1
        }
    }

    % Define how \section marks the header
    \renewcommand{\sectionmark}[1]{
        \markright{
            \ifnum \c@secnumdepth > \z@
                \thesection\ \ %
            \fi
            ##1
        }
    }
}

% ---------------------------------------------------------------------
% Page Style: "plainheader"
% ---------------------------------------------------------------------
% A simpler page style with just the page number in the header

\newcommand{\ps@plainheader}{
    \renewcommand{\@oddfoot}{}
    \renewcommand{\@evenfoot}{}
    \renewcommand{\@oddhead}{\hfill{\thePageFormat\thepage}}
    \renewcommand{\@evenhead}{{\thePageFormat\thepage}\hfill}

    % Ensure \chapter* shows the header
    \let\@mkboth\customMarkBoth

    % Define how \chapter marks the header
    \renewcommand{\chaptermark}[1]{
        \markboth{
            \ifnum \c@secnumdepth > \m@ne
                \thechapter\ \ %
            \fi ##1
        }{
            \ifnum \c@secnumdepth > \m@ne
                \thechapter\ \ %
            \fi ##1
        }
    }

    % Define how \section marks the header
    \renewcommand{\sectionmark}[1]{
        \markright{
            \ifnum \c@secnumdepth > \z@
                \thesection\ \ %
            \fi ##1
        }
    }
}

% ---------------------------------------------------------------------
% Page Style: "ruledheader"
% ---------------------------------------------------------------------
% A page style with underlined headers for both odd and even pages

\newcommand{\ps@ruledheader}{
    \renewcommand{\@oddfoot}{}
    \renewcommand{\@evenfoot}{}
    \renewcommand{\@oddhead}
    {
        \underline{
            \makebox[\textwidth]{
                \raisebox{-.5ex}{}
                {\rightMarkFormat\rightmark}\hfill{\thePageFormat\thepage}
            }
        }
    }
    \renewcommand{\@evenhead}
    {
        \underline{
            \makebox[\textwidth]{
                \raisebox{-.5ex}{}
                {\thePageFormat\thepage}\hfill{\leftMarkFormat\leftmark}
            }
        }
    }

    % Ensure \chapter* shows the header
    \let\@mkboth\customMarkBoth

    % Define how \chapter marks the header
    \renewcommand{\chaptermark}[1]{
        \markboth{
            \ifnum \c@secnumdepth > \m@ne
                \thechapter\ \ %
            \fi ##1
        }{
            \ifnum \c@secnumdepth > \m@ne
                \thechapter\ \ %
            \fi ##1
        }
    }

    % Define how \section marks the header
    \renewcommand{\sectionmark}[1]{
        \markright{
            \ifnum \c@secnumdepth > \z@
                \thesection\ \ %
            \fi ##1
        }
    }
}

% ---------------------------------------------------------------------
% Page Style: "fancyPlain"
% ---------------------------------------------------------------------
% A page style with underlined headers for both odd and even pages
% after the title page

\newcommand{\ps@fancyPlain}{
    \lhead{\fancyplain{}{\textit{\leftmark}}}
    \chead{}
    \rhead{\fancyplain{}{\thepage}}
    \lfoot{}
    \cfoot{}
    \rfoot{}
    \pagestyle{fancyplain}
}

% ---------------------------------------------------------------------
% Indenting First Paragraph of Each Section
% ---------------------------------------------------------------------

% Conditionally load the indentfirst package if the first paragraph should be indented
\ifthenelse{\boolean{indentFirstParagraph}}
{\RequirePackage{indentfirst}}
{}

% Set the paragraph indentation size and spacing between paragraphs
\setlength{\parindent}{\indentationSize}
\setlength{\parskip}{\indentationSkip}

% Set the default page style for chapter titles
\newcommand{\chapterTitlePageStyle}{plain}
